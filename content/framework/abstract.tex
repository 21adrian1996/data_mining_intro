\chapter*{Abstract\markboth{Abstract}{}}
Diese Arbeit bietet einen Überblick über die Thematiken im Bereich Data Mining. Dazu werden einleitend Beispiele vorgestellt, bei welchen Data Mining verwendet wurde. Hierbei handelt es sich einerseits um Beispiele aus dem Bereich Gesundheit, Sicherheit, Landwirtschaft sowie ein kommerzielles Beispiel.
Zudem wird in der Arbeit eine kurze Übersicht zu den ethischen Überlegungen und Bedenken gegeben.
Der Hauptteil der Arbeit beschäftigt sich mit den verschiedenen Verfahren im Bereich Data Mining und Machine Learning.

Dabei soll eine Übersicht über die Verfahren, sowie die wichtigsten Stichwörter dazu gegeben werden. Diese Übersicht soll es der Leserschaft ermöglichen einen Überblick zu den Verfahren zu erhalten, aber auch gezielt weiterführende Themen ansprechen. Somit soll die Leserschaft gezielt motiviert werden, die weiterführenden Themen nach eigenen Interessen durch die referenzierten Dokumente zu vertiefen.

Gegen Ende des Dokumentes werden konkret die Verfahren Random Forest und K-means vorgestellt. Diese dienen beispielhaft dazu, die Komplexität des Data Mining aufzuzeigen. Auch hier werden gezielt Stichwörter eingesetzt, welche die Leserschaft für eigene, weitere Recherchen verwenden kann.
