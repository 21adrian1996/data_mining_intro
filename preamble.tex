% ##################################################
% Document variables
% ##################################################

% Personal data of the author
\newcommand{\docSurnameA}{Berger}
\newcommand{\docPrenameA}{Adrian}
\newcommand{\docEmailA}{adrian.berger.2@students.bfh.ch}

% Data of the University
\newcommand{\docLocationFH}{Bern}
\newcommand{\docFieldOfStudy}{BSc Computer Science}


% Document data
\newcommand{\docTitle}{Data Mining}
\newcommand{\docSecondTitle}{Einführung und Beispiele}
\newcommand{\docKindOfWork}{IT-Seminar}
\newcommand{\docHandOverDate}{25.05.2021}
\newcommand{\docFirstLector}{Jürgen Eckerle}
\newcommand{\docModule}{BTI7311}

% Used for toggle-if case
\usepackage{etoolbox}

% Conditional variables
\providetoggle{useCode}
\settoggle{useCode}{false}
\providetoggle{useAbstract}
\settoggle{useAbstract}{true}

% ##################################################
% General packages
% ##################################################

\usepackage{cmap} % Make PDF files searchable and copyable
\usepackage[T1]{fontenc} % used for west European language support
\usepackage[utf8]{inputenc} % Allows usage of umlaute
\usepackage{multicol}

% Set language to english
\usepackage[german]{babel}

% Use graphics/pictures
\usepackage{graphicx}

% Colours (be able to define custom colours)
\usepackage{color}
\usepackage[usenames,dvipsnames,svgnames,table]{xcolor}

% Masking of URLs and file paths
\usepackage[hyphens]{url}

% Nice quotes
\usepackage{csquotes}

% Package for indexing (Schlagwortverzeichnis)
%\usepackage{index}
%\makeindex

% Ipsum Lorem
\usepackage{lipsum}


% ##################################################
% Page formatting
% ##################################################
\usepackage[
	inner=2.5cm, % Left margin
	outer=2.5cm, % Right marin
	top=1.5cm,   % Top marin
	bottom=2cm,  % Bottom marin
	includefoot, % Include footer to bottom margin
	includehead  % Include header to top margin
	]{geometry}

% ##################################################
% Header and footer
% ##################################################

\usepackage{fancyhdr}  % allows manipulation of header and footer

\pagestyle{fancy} % lets you define your own style
\fancyhf{}
\fancyhead[EL,OR]{\sffamily\thepage} % page number even=left, odd=right
\fancyhead[ER,OL]{\sffamily\nouppercase{\leftmark}} % chapter even=left, odd=right
\fancyfoot[LE,LO]{Berner Fachhochschule}
\fancyfoot[RE,RO]{\docPrenameA~\docSurnameA}


\renewcommand{\footrulewidth}{1pt}		% add footer line by setting it to one

\fancypagestyle{headings}{}
\fancypagestyle{plain}{}

% No header/footer on empty pages
\fancypagestyle{empty}{
  \fancyhf{}
  \renewcommand{\headrulewidth}{0pt}
  \renewcommand{\footrulewidth}{0pt}
}


%Saves \chaptermark in \oldchaptermark so that 
% it can be reset for the appendix
\let\oldchaptermark\chaptermark

%No "Chapter # NAME" in header
\renewcommand{\chaptermark}[1]{
	\markboth{#1}{}
   	\markboth{\thechapter.\ #1}{}
}

% ##################################################
% fonts
% ##################################################

% Set default font
\renewcommand{\familydefault}{\sfdefault}

% Set default line distance to 1.5
\usepackage{setspace}
\onehalfspacing 

% Set font size
\addtokomafont{chapter}{\sffamily\Large\bfseries} 
\addtokomafont{section}{\sffamily\large\bfseries} 
\addtokomafont{subsection}{\sffamily\normalsize\bfseries} 
\addtokomafont{caption}{\sffamily\normalsize\mdseries} 

%Disable indent of paragraphs
\setlength{\parindent}{0pt}

%Line distances of paragraphs
\usepackage{parskip}

% ##################################################
% Table of contents / General listings
% ##################################################

% Control table of contents, figures
\usepackage{tocloft}

% Points also for chapters
\renewcommand{\cftchapdotsep}{3}
\renewcommand{\cftdotsep}{3}

% Adjust font and size in table of contents
\renewcommand{\cftchapfont}{\sffamily\normalsize}
\renewcommand{\cftsecfont}{\sffamily\normalsize}
\renewcommand{\cftsubsecfont}{\sffamily\normalsize}
\renewcommand{\cftchappagefont}{\sffamily\normalsize}
\renewcommand{\cftsecpagefont}{\sffamily\normalsize}
\renewcommand{\cftsubsecpagefont}{\sffamily\normalsize}

%Set space between lines in listings
\setlength{\cftparskip}{.5\baselineskip}
\setlength{\cftbeforechapskip}{.1\baselineskip}


% ##################################################
% List of tables and tables
% ##################################################

% Numbering of tables
\renewcommand{\thetable}{\arabic{table}}
\counterwithout{table}{chapter}

% Adjust list of tables
\renewcommand{\cfttabpresnum}{Table }
\renewcommand{\cfttabaftersnum}{:}

% Width of numbering scope [Table 1:]
\newlength{\tableLength}
\settowidth{\tableLength}{\bfseries\cfttabpresnum\cfttabaftersnum}
\addtolength{\tableLength}{3mm} %extra offset
\setlength{\cfttabnumwidth}{\tableLength}
\setlength{\cfttabindent}{0cm}

%Adjust font
\renewcommand\cfttabfont{\sffamily}
\renewcommand\cfttabpagefont{\sffamily}

% Suppress vertical lines
\usepackage{booktabs}

%Multi row for specific rows
\usepackage{multirow}

%Additional table package
\usepackage{tabu}

% ##################################################
% Table of figures and figures
% ##################################################

\usepackage{caption}

\usepackage{wrapfig}

% Numbering of figures
\renewcommand{\thefigure}{\arabic{figure}}
\usepackage{chngcntr}
\counterwithout{figure}{chapter}

% Adjust table of figures
\renewcommand{\cftfigpresnum}{Figure }
\renewcommand{\cftfigaftersnum}{:}

% Width of numbering scope [Figure 1:]
\newlength{\figureLength}
\settowidth{\figureLength}{\bfseries\cftfigpresnum\cftfigaftersnum}
\addtolength{\figureLength}{2mm} %extra offset
\setlength{\cftfignumwidth}{\figureLength}
\setlength{\cftfigindent}{0cm}

% Adjust font
\renewcommand\cftfigfont{\sffamily}
\renewcommand\cftfigpagefont{\sffamily}

%Default paths
\graphicspath{ {./content/pictures/} }


% ##################################################
% Appendix
% ##################################################

%Calc packet for calculations
\usepackage{calc}
\usepackage{amsmath}

%Appendix packet, set the flags for the TOC
\usepackage[toc,title,titletoc]{appendix} 


% Change text for title
%\renewcommand{\appendixtocname}{Appendix}

%Befehl für einen neuen Bericht und die erste Seite als Bild
\newcommand{\appendixsection}[2]{
\section{#1}
\appendixsingle{#2}
}

%Befehl für einzelne Seite als Bild eingefasst, damit Überschrift und Kopfzeile
% bestehen bleibt. 
\newcommand{\appendixsingle}[1]{
\vspace{-10cm}
\vfill
\mbox{}\hspace{-1.5cm}\includegraphics[width=\linewidth+3cm]{#1}\hspace{-1.5cm}\mbox{}
\vspace{-10cm}
\vfill
\mbox{}
}

%Datenträger Tabelle
\definecolor{lightgray}{gray}{0.85}
\definecolor{ultralightgray}{gray}{0.95}
\definecolor{mygray}{gray}{0.70}

% ##################################################
% Theoreme
% ##################################################

% Umgebung fuer Beispiele
%\newtheorem{beispiel}{Beispiel}

% Umgebung fuer These
%\newtheorem{these}{These}

% Umgebung fuer Definitionen
%\newtheorem{definition}{Definition}
  	
% ##################################################
% Literaturverzeichnis
% ##################################################
\usepackage[sorting=none]{biblatex}
\bibliography{sample} % file to use (without .bib)
%\bibliographystyle{alpha} % set style of bibliography

% ##################################################
% Abkuerzungsverzeichnis
% ##################################################

\usepackage{acronym}

% ##################################################
% PDF / Dokumenteninternelinks
% ##################################################

\usepackage[
	colorlinks=false,
   	linkcolor=black,
   	citecolor=black,
  	filecolor=black,
	urlcolor=black,
    bookmarks=true,
    bookmarksopen=true,
    bookmarksopenlevel=3,
    bookmarksnumbered,
    plainpages=false,
    pdfpagelabels=true,
    hyperfootnotes,
    hidelinks,
    pdftitle ={\docTitle},
    pdfauthor={\docPrenameA~\docSurnameA},
    pdfcreator={\docPrenameA~\docSurnameA}]{hyperref}

